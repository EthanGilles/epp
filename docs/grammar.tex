%+++++++++++++++++++++++++++++++++++++++++++++++++++++++++++
% Ethan Gilles (https://github.com/EthanGilles)
% Homework Template
% +++++++++++++++++++++++++++++++++++++++++++++++++++++++++++

% Edit 'chapsec', 'yourname', and 'course'
\def\filename{}   		   % included file (edit file)
\def\chapsec{Formal Grammar}  % Chapter/Section/Topic
\def\yourname{Ethan Gilles}	   % Your name
\def\course{}		   % Course (if different)

% -----------------------------------------------------------
\documentclass[11pt]{article}

\def\pf{\textit{Proof}: }

\usepackage{mathtools}
\usepackage{epsfig}
\usepackage{amsfonts}
\usepackage{amssymb}
\usepackage{amstext}
\usepackage{amscd}
\usepackage{amsmath}
\usepackage{xspace}
\usepackage{theorem}
\usepackage{float}
\usepackage[table]{xcolor}
\usepackage{color}
\usepackage{soul}
\usepackage{booktabs}
\usepackage{outlines}
\usepackage{enumitem}
\setenumerate[1]{label=\arabic*.}
\setenumerate[2]{label=(\alph*).}
\setenumerate[3]{label=\roman*.}
\setenumerate[4]{label=\alph*.}

\usepackage{hyperref}
\hypersetup{
    colorlinks=true,
    linkcolor=blue,
    filecolor=magenta,      
    urlcolor=cyan,
    pdftitle={Overleaf Example},
    pdfpagemode=FullScreen,
    }

% TikZ
\usepackage{tikz}
\usepackage{pgfplots}
\pgfplotsset{compat=1.15}
\usepackage{mathrsfs}
\usetikzlibrary{arrows}

% Colors
\definecolor{stainlessSteel}{cmyk}{0,0,0.02,0.12}

% Document Geometry
\makeatletter
 \setlength{\textwidth}{6.75in}
 \setlength{\oddsidemargin}{0in}
 \setlength{\evensidemargin}{0in}
 \setlength{\topmargin}{0.0125in}
 \setlength{\textheight}{9.0in}
 \setlength{\headheight}{0pt}
 \setlength{\headsep}{0pt}
 \setlength{\marginparwidth}{59pt}

 \setlength{\parindent}{0pt}
 \setlength{\parskip}{5pt plus 1pt}
 \setlength{\theorempreskipamount}{5pt plus 1pt}
 \setlength{\theorempostskipamount}{0pt}
 \setlength{\abovedisplayskip}{8pt plus 3pt minus 6pt}
 \setlength{\intextsep}{15pt plus 3pt minus 6pt}

 % Headings
 \renewcommand{\section}{\@startsection{section}{1}{0mm}%
    {2ex plus -1ex minus -.2ex}%
    {1.3ex plus .2ex}%
    {\normalfont\Large\bfseries}}%
 \renewcommand{\subsection}{\@startsection{subsection}{2}{0mm}%
    {1ex plus -1ex minus -.2ex}%
    {1ex plus .2ex}%
    {\normalfont\large\bfseries}}%
 \renewcommand{\subsubsection}{\@startsection{subsubsection}{3}{0mm}%
    {1ex plus -1ex minus -.2ex}%
    {1ex plus .2ex}%
    {\normalfont\normalsize\bfseries}}
 \renewcommand\paragraph{\@startsection{paragraph}{4}{0mm}%
    {1ex \@plus1ex \@minus.2ex}%
    {-1em}%
    {\normalfont\normalsize\bfseries}}
 \renewcommand\subparagraph{\@startsection{subparagraph}{5}{\parindent}%
    {2.0ex \@plus1ex \@minus .2ex}%
    {-1em}%
    {\normalfont\normalsize\bfseries}}
\makeatother

\newcounter{thelecture}

\newenvironment{proof}{{\bf Proof:  }}{\hfill\rule{2mm}{2mm}}
\newenvironment{proofof}[1]{{\bf Proof of #1:  }}{\hfill\rule{2mm}{2mm}}
\newenvironment{proofofnobox}[1]{{\bf#1:  }}{}
\newenvironment{example}{{\bf Example: }}{\hfill\rule{0mm}{0mm}} % change 2mm 2mm for square

%\renewcommand{\theequation}{\thesection.\arabic{equation}}
%\renewcommand{\thefigure}{\thesection.\arabic{figure}}

\newtheorem{fact}{Fact}
\newtheorem{lemma}[fact]{Lemma}
\newtheorem{theorem}[fact]{Theorem}
\newtheorem{definition}[fact]{Definition}
\newtheorem{corollary}[fact]{Corollary}
\newtheorem{proposition}[fact]{Proposition}
\newtheorem{claim}[fact]{Claim}
\newtheorem{exercise}[fact]{Exercise}

% math notation
\newcommand{\R}{\ensuremath{\mathbb R}}
\newcommand{\Z}{\ensuremath{\mathbb Z}}
\newcommand{\N}{\ensuremath{\mathbb N}}
\newcommand{\B}{\ensuremath{\mathbb B}}
\newcommand{\F}{\ensuremath{\mathcal F}}
\newcommand{\SymGrp}{\ensuremath{\mathfrak S}}
\newcommand{\prob}[1]{\ensuremath{\text{{\bf Pr}$\left[#1\right]$}}}
\newcommand{\expct}[1]{\ensuremath{\text{{\bf E}$\left[#1\right]$}}}
\newcommand{\size}[1]{\ensuremath{\left|#1\right|}}
\newcommand{\ceil}[1]{\ensuremath{\left\lceil#1\right\rceil}}
\newcommand{\floor}[1]{\ensuremath{\left\lfloor#1\right\rfloor}}
\newcommand{\ang}[1]{\ensuremath{\langle{#1}\rangle}}
\newcommand{\poly}{\operatorname{poly}}
\newcommand{\polylog}{\operatorname{polylog}}

% Anupam's abbreviations
\newcommand{\e}{\epsilon}
\newcommand{\half}{\ensuremath{\frac{1}{2}}}
\newcommand{\junk}[1]{}
\newcommand{\sse}{\subseteq}
\newcommand{\union}{\cup}
\newcommand{\meet}{\wedge}
\newcommand{\dist}[1]{\|{#1}\|_{\text{dist}}}
\newcommand{\hooklongrightarrow}{\lhook\joinrel\longrightarrow}
\newcommand{\embeds}[1]{\;\lhook\joinrel\xrightarrow{#1}\;}
\newcommand{\mnote}[1]{\normalmarginpar \marginpar{\tiny #1}}



% -----------------------------------------------------------
% Header
\newcommand{\hwheadings}[3]{
{{\bf Epp}: \chapsec } \hfill {{ \yourname }} \hfill {{ \course #1}}
\rule[0.051in]{\textwidth}{0.0025in}
% \thispagestyle{empty}
}

% Document begins here 
\begin{document}
\hwheadings{}{}{}

\textbf{Project Idea} \\
Create a programming language and compiler. Below are some formal grammar rules 
and syntax for the language, E++. \\

\textbf{Syntax}
\begin{itemize}
  \item Variable \textit{shadowing} is \textbf{not} allowed. Having a variable with the same name 
    in an inner scope is invalid, unless the variable is first 
    declared in the inner scope. 
  \item You must terminate a statement with a semicolon, \textit{unless} that 
    statement \textbf{ends} with a scope. 
  \item Single line comments are denoted by `//' and Multi-line comments are denoted with `/* */'
  \item You must say \textbf{please} or \textbf{PLEASE} enough to satisfy the compiler.
\end{itemize}

\textbf{Producers}
\begin{align*}
  [\text{Program}] &\to [\text{Statement}]^* \\
  [\text{Statement}] &\to 
  \begin{cases}
    \text{please} \\
    \text{PLEASE} \\
    \text{exit}([\text{Expr}]); \\ 
    \text{set} \; \text{ID} = [\text{Expr}]; \\
    \text{reset} \; \text{ID} = [\text{Expr}]; \\
    \text{if} ([\text{Expr}])[\text{Scope}][\text{AfterIf}]  \\
    [\text{Scope}] \\
  \end{cases} \\
  [\text{Scope}] &\to \{[\text{Stmt}]^*\} \\
  [\text{AfterIf}] &\to 
  \begin{cases}
    \text{elsif}([\text{Expr}])\text{[Scope]}\text{[AfterIf]} \\
    \text{else}([\text{Expr}])\text{[Scope]} \\
    \epsilon
  \end{cases} \\
  [\text{Expr}] &\to 
  \begin{cases}
  [\text{Term}] \\ 
  [\text{BinaryExpr}] \\
  \end{cases} \\
  [\text{Term}] &\to
  \begin{cases}
    \text{int\_lit} \\ 
    \text{ID} \\ 
    \left([\text{Expr}]\right)
  \end{cases} \\
\end{align*}

\begin{align*}
  [\text{BinaryExpr}] &\to 
  \begin{cases}
    [\text{Expr}] \times [\text{Expr}] & precedence = 2 \\
    [\text{Expr}] \div [\text{Expr}] & precedence = 2 \\
    [\text{Expr}] \,\%\, [\text{Expr}] & precedence = 2 \\
    [\text{Expr}] + [\text{Expr}] & precedence = 1 \\
    [\text{Expr}] - [\text{Expr}] & precedence = 1 \\
    [Expr] \le  [Expr] & precedence = 0 \\
    [Expr] \ge  [Expr] & precedence = 0 \\
    [Expr] >  [Expr] & precedence = 0 \\
    [Expr] <  [Expr] & precedence = 0 \\
    [Expr] ==  [Expr] & precedence = 0 \\
    [Expr] \; ! \!=  [Expr] & precedence = 0 \\
  \end{cases} \\
\end{align*}

\textbf{Terminals}
\begin{align*}
  \text{int\_lit} &\to \text{[0-9]}^* \\
  \text{ID} &\to \text{[a-zA-Z][a-zA-Z0-9]}^* \\
\end{align*}

\end{document}
% -----------------------------------------------------------
